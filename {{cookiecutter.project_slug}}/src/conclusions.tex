\section{Conclusions}\label{sec:conclusions}

The methods presented above are implemented in the Python package \texttt{bayesian\_pyhf}~\cite{BayesianPyhf}.
This software package enables the parallel Bayesian and frequentist analysis of multi-channel binned models within the single software framework \texttt{pyhf}.
The current interface of the package \texttt{bayesian\_pyhf} is demonstrated in Listing~\ref{lst:pymc_example}.
Further enhancements regarding the user interface and stability with respect to multi-chain sampling are ongoing.
A full integration in the \texttt{pyhf} library is also planned.

\begin{listing}
 \inputminted{python}{src/code/pymc_example.py}
 \caption{Pseudo-code for evaluating \texttt{HistFactory} models (\texttt{model}) using \texttt{PyMC} given unconstrained parameters (\texttt{unconstr\_priors}) and observations (\texttt{data}). \texttt{post(prior)\_pred} are the posterior (prior) predictives and \texttt{post\_data} are the samples from the posterior distribution. Following the \texttt{PyMC} syntax~\cite{PyMC}, the \texttt{with} statement opens a context, that initializes the inference in a way that all actions within the block are interpreted with respect to the given model, data and priors. In addition, the methodologies regarding conjugate priors from Sec.~\ref{subsec:HFandpyhf} are applied under the hood, resulting in the constraint priors which are added to the model parameters for sampling.}
 \label{lst:pymc_example}
\end{listing}
